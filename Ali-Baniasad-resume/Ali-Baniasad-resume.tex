%-------------------------
% Resume in Latex
% Author : Jake Gutierrez
% Based off of: https://github.com/sb2nov/resume
% License : MIT
%------------------------

\documentclass[letterpaper,11pt]{article}

\usepackage{latexsym}
\usepackage[empty]{fullpage}
\usepackage{titlesec}
\usepackage{marvosym}
\usepackage[usenames,dvipsnames]{color}
\usepackage{verbatim}
\usepackage{enumitem}
\usepackage[hidelinks]{hyperref}
\usepackage{fancyhdr}
\usepackage[english]{babel}
\usepackage{tabularx}
\usepackage{fontawesome5}
\usepackage{multicol}
\setlength{\multicolsep}{-3.0pt}
\setlength{\columnsep}{-1pt}
\input{glyphtounicode}
\newcommand{\J}{\textbf{[J]}\ }  % journals
\newcommand{\C}{\textbf{[C]}\ }  % conferences
\newcommand{\etal}{\textit{et al.}}
\usepackage{ragged2e}   % gives \justifying


%\hypersetup{
%	colorlinks,
%	citecolor=violet,
%	linkcolor=red,
%	urlcolor=blue}

%----------FONT OPTIONS----------
% sans-serif
% \usepackage[sfdefault]{FiraSans}
% \usepackage[sfdefault]{roboto}
% \usepackage[sfdefault]{noto-sans}
% \usepackage[default]{sourcesanspro}

% serif
% \usepackage{CormorantGaramond}
% \usepackage{charter}


\pagestyle{fancy}
\fancyhf{} % clear all header and footer fields
\fancyfoot{}
\renewcommand{\headrulewidth}{0pt}
\renewcommand{\footrulewidth}{0pt}

% Adjust margins
\addtolength{\oddsidemargin}{-0.6in}
\addtolength{\evensidemargin}{-0.5in}
\addtolength{\textwidth}{1.19in}
\addtolength{\topmargin}{-.7in}
\addtolength{\textheight}{1.4in}

\urlstyle{same}

\raggedbottom
\raggedright
\setlength{\tabcolsep}{0in}

% Sections formatting
\titleformat{\section}{
  \vspace{-4pt}\scshape\raggedright\large\bfseries
}{}{0em}{}[\color{black}\titlerule \vspace{-5pt}]

% Ensure that generate pdf is machine readable/ATS parsable
\pdfgentounicode=1

%-------------------------
% Custom commands
\newcommand{\resumeItem}[1]{
  \item\small{
    {#1 \vspace{-2pt}}
  }
}

\newcommand{\classesList}[4]{
    \item\small{
        {#1 #2 #3 #4 \vspace{-2pt}}
  }
}

\newcommand{\resumeSubheading}[4]{
  \vspace{-2pt}\item
    \begin{tabular*}{1.0\textwidth}[t]{l@{\extracolsep{\fill}}r}
      \textbf{ #1} & \textbf{\small #2} \\
      \textsl{\small #3} & \textit{\small #4} \\
    \end{tabular*}\vspace{-7pt}
}

\newcommand{\resumeSubSubheading}[2]{
    \item
    \begin{tabular*}{0.97\textwidth}{l@{\extracolsep{\fill}}r}
      \textit{\small#1} & \textit{\small #2} \\
    \end{tabular*}\vspace{-7pt}
}

\newcommand{\resumeProjectHeading}[2]{
    \item
    \begin{tabular*}{1.001\textwidth}{l@{\extracolsep{\fill}}r}
      \small#1 & \textbf{\small #2}\\
    \end{tabular*}\vspace{-7pt}
}

\newcommand{\resumeSubItem}[1]{\resumeItem{#1}\vspace{-4pt}}

\renewcommand\labelitemi{$\vcenter{\hbox{\tiny$\bullet$}}$}
\renewcommand\labelitemii{$\vcenter{\hbox{\tiny$\bullet$}}$}

\newcommand{\resumeSubHeadingListStart}{\begin{itemize}[leftmargin=0.0in, label={}]}
\newcommand{\resumeSubHeadingListEnd}{\end{itemize}}
\newcommand{\resumeItemListStart}{\begin{itemize}}
\newcommand{\resumeItemListEnd}{\end{itemize}\vspace{-5pt}}

%-------------------------------------------
%%%%%%  RESUME STARTS HERE  %%%%%%%%%%%%%%%%%%%%%%%%%%%%


\begin{document}

%----------HEADING----------
% \begin{tabular*}{\textwidth}{l@{\extracolsep{\fill}}r}
%   \textbf{\href{http://sourabhbajaj.com/}{\Large Sourabh Bajaj}} & Email : \href{mailto:sourabh@sourabhbajaj.com}{sourabh@sourabhbajaj.com}\\
%   \href{http://sourabhbajaj.com/}{http://www.sourabhbajaj.com} & Mobile : +1-123-456-7890 \\
% \end{tabular*}
\begin{center}
    {\Huge \scshape Ali BaniAsad} \\ \vspace{1pt}
    Tehran, Iran \\ \vspace{1pt}
    \small 
    \href{https://alibaniasad1999.github.io}{\raisebox{-0.\height}\faGlobe\ \textcolor{blue}{Personal Website}}  ~
    \href{https://linkedin.com/in/alibaniasad1999/}{\raisebox{-0.\height}\faLinkedin\ \textcolor{blue}{alibaniasad1999}}  ~
    \href{https://github.com/alibaniasad1999}{\raisebox{-0.\height}\faGithub\ \textcolor{blue}{alibaniasad1999}} ~ 
    \href{https://scholar.google.com/citations?user=KeKpSvEAAAAJ}{\raisebox{-0.\height}\faGraduationCap\ \textcolor{blue}{Scholar}} ~
    \href{mailto:alibaniasad1999@yahoo.com}{\raisebox{-0.\height}\faEnvelope\  \textcolor{blue}{{alibaniasad1999@yahoo.com}}} 
    \vspace{-8pt}
\end{center}

\vspace{-8pt}
%-----------EDUCATION-----------
\section{Education}
  \resumeSubHeadingListStart

  \resumeSubheading
  {Sharif University of Technology}{Sep.\ 2022 -- Apr.\ 2025
  }
  {Master of Science in Aerospace Engineering
  % , GPA 3.22 / 4.00  (Top 10\%)
  }{Tehran, Iran}
  \vspace{-3pt}

    \resumeSubheading
      {Sharif University of Technology}{Sep.\ 2017 -- May\ 2022
      }
      {Bachelor of Science in Aerospace Engineering
      % GPA 3.46/4.00; Upper-division 3.72/4.00
      }{Tehran, Iran}

%      \resumeSubheading
%      {Allameh Helli High School
%      }{Sep. 2010 -- July 2017}
%      {Diploma in Mathematics,
%      GPA: 4/4 (19.85/20)
%      }{Kerman, Iran}


  \resumeSubHeadingListEnd


 \vspace{-16pt}
%------RELEVANT COURSEWORK-------
\section{Research Interests}
    %\resumeSubHeadingListStar
        \begin{multicols}{3}
          {\normalsize   
            \begin{itemize}[itemsep=-2pt, parsep=3pt]
                \item Reinforcement Learning
                \item Optimal Control
                % \item Artificial Intelligence
                \item Deep Learning
                \item Robotics
                % \item Computer Vision
                \item Automatic Control
                \item Game Theory
            \end{itemize}}
        \end{multicols}
        \vspace*{3.0\multicolsep}
    %\resumeSubHeadingListEnd

\vspace{4pt}
    \section{Publications
    % \small
    % \href{https://scholar.google.com/citations?user=KeKpSvEAAAAJ&hl=en}{{\normalfont \textit{\textcolor{blue}{[Google Scholar profile \faGraduationCap]}}}}
    }
    % \subsection*{Journal}
    %\resumeSubHeadingListStartt
    %\resumeSubheading{Fraternity}{Spring 2020 -- Present}{President}{University Name}
    \resumeItemListStart
    % \justifying
    \resumeItem{2025 \J Sharifi, A., \textbf{BaniAsad, A.} \etal\ “Applied an In-Motion Transfer Alignment Approach During Global Positioning System Outages Utilizing a Recurrent Neural Network Algorithm.” \textit{Eng.\ Appl.\ AI} — \href{https://github.com/alibaniasad1999/INS_AI/blob/main/Report/revised.pdf}{\textcolor{blue}{minor review}}.}
    \vspace{-4pt}
    \resumeItem{2025 \C \textbf{BaniAsad, A.}, Nobahari, H. “Robust DDPG Reinforcement Learning Differential Game Guidance in Low-Thrust, Multi-Body Dynamical Environments.” \textit{23rd Int. Conf. of Iranian Aerospace Society} — Accepted.}
    \vspace{-4pt}
    \resumeItem{2025 \C Amirpour, M., \textbf{BaniAsad, A.}, Nobahari, H. “Reinforcement Learning-Based Controller Design for a Suspended Ball Plant.” \textit{23rd Int. Conf. of Iranian Aerospace Society} — Accepted.}
    \vspace{-4pt}
    \resumeItem{2024 \J \textbf{BaniAsad, A.} \etal\ “Attitude Control of a 3-DoF Quadrotor Platform Using a Linear Quadratic Integral Differential Game Approach.” \textit{ISA Trans.} — \href{https://doi.org/10.1016/j.isatra.2024.03.005}{\textcolor{blue}{Elsevier DOI}}.}
    \vspace{-4pt}
    \resumeItem{2022 \C Nobahari, H., \textbf{BaniAsad, A.} \etal\ “Linear Quadratic Integral Differential Game Applied to the Real-Time Control of a Quadrotor Experimental Setup.” \textit{ICRoM} — \href{https://doi.org/10.1109/ICRoM57054.2022.10025263}{\textcolor{blue}{IEEE DOI}}.}









    % \resumeItem{2025 \J \textbf{BaniAsad, A.}, Sharifi, A.\ \etal\ “Applied an In-Motion Transfer Alignment Approach During Global Positioning System Outages Utilizing a Recurrent Neural Network Algorithm.” \textit{Eng.\ Appl.\ AI} — \href{https://drive.google.com/file/d/19qd1IjmCQMv2Alldh55oYQSTPr03vjJL/view?usp=sharing}{\textcolor{blue}{minor review}}.}
    % \resumeItem{Sharifi, A., \textbf{BaniAsad, A.} \etal\ “Applied an In-Motion Transfer Alignment Approach During GPS Outages via a Recurrent Neural-Network Algorithm.” \textit{Engineering Applications of Artificial Intelligence}, 2025 — \href{https://drive.google.com/file/d/19qd1IjmCQMv2Alldh55oYQSTPr03vjJL/view?usp=sharing}{\textcolor{blue}{minor review}}.}
    % \resumeItem{Sharifi, A., \textbf{BaniAsad, A.} \etal\ “Applied an In-Motion Transfer Alignment Approach During Global Positioning System Outages Utilizing a Recurrent Neural Network Algorithm.” \textit{Engineering Applications of Artificial Intelligence}, 2025 — \href{https://drive.google.com/file/d/19qd1IjmCQMv2Alldh55oYQSTPr03vjJL/view?usp=sharing}{\textcolor{blue}{minor review}}.}

    %  \vspace{-6pt}
    % \resumeItem{\textbf{Ali BaniAsad}, \J Reza Pordal, Alireza Sharifi, Hadi Nobahari. "Attitude Control of a 3-DoF Quadrotor Platform Using a Linear Quadratic Integral Differential Game Approach." \textit{ISA Transactions,
    %     \href{https://doi.org/10.1016/j.isatra.2024.03.005}{\textcolor{blue}{Elsevier}}}, 2024.}
    %     \vspace{-6pt}
    %     \resumeItemListEnd
    %  \vspace{-12pt}
    %   \subsection*{Conference}
    %  \vspace{-6pt}
    %     \resumeItemListStart
    %     \resumeItem{\textbf{Ali BaniAsad} and Hadi Nobahari. "Robust DDPG Reinforcement Learning Differential Game Guidance in Low-Thrust, Multi-Body Dynamical Environments." \textit{The 23rd International Conference of Iranian Aerospace Society}, 2025. (Accepted)}
    %     \vspace{-6pt}
    %     \resumeItem{Mahdi Amirpour,  \textbf{Ali BaniAsad}, Hadi Nobahari. "Reinforcement Learning-Based Controller Design for a Suspended Ball Plant" \textit{The 23rd International Conference of Iranian Aerospace Society}, 2025. (Accepted)}
    %     \vspace{-6pt}
    % \resumeItem{Hadi Nobahari, \etal\ \textbf{Ali BaniAsad}, Alireza Sharifi. "Linear Quadratic Integral Differential Game Applied to the Real-time Control of a Quadrotor Experimental Setup." \textit{ICRoM, \href{https://doi.org/10.1109/ICRoM57054.2022.10025263}{\textcolor{blue}{IEEE}}}, 2022.}
    %  %\vspace{-6pt}
    %\resumeItem{Alireza Sharifi, \textbf{Ali BaniAsad}. "Robust In-Motion Transfer Alignment of Low-Grade Inertial Navigation Systems with Recurrent Neural Networks in the Event of Reference Malfunction." \textit{IEEE}, 2024 (Active)}
    % \vspace{-6pt}
    %\resumeItem{\textbf{Ali BaniAsad}, Hadi Nobahari. "Robust Differential Game Reinforcement Learning with Soft Actor-Critic for Guidance in Low-Thrust Multi-Body Environments." \textit{AIAA}, 2024 (Active)}
    \resumeItemListEnd
    %\resumeSubHeadingListEnd
    \vspace{-12pt}

%-----------EXPERIENCE-----------
\section{Research Experience \& Projects}
  \resumeSubHeadingListStart

  \resumeSubheading
  {Embedded RL Control for Robots on Resource-Constrained Hardwares%
    {\href{https://github.com/alibaniasad1999/master-thesis}{ \faGithub}}%
  }{Aug.\ 2022 -- Apr.\ 2025}
  {Master’s Thesis, Sharif University of Technology}{Tehran, Iran}
  \resumeItemListStart
% \resumeItem{Built a 15 k-LOC PyTorch/TensorFlow + Gymnasium stack for real-time control on embedded robots.}%
% \resumeItem{Implemented four RL algorithms: DDPG, TD3, SAC, and PPO.}%
% \resumeItem{Outperformed classical MPC, cutting trajectory-tracking error by 22\%  within strict on-board CPU/memory limits.}%
%     % \resumeItem{Introduced disturbance-augmented adversarial training; policies maintained stable guidance under $10\times$ worst-case perturbations in evaluation sweeps.}%
%     % \resumeItem{Built a mixed-fidelity stack spanning Python, MATLAB/Simulink, and C++.}%
%     \resumeItem{Crafted zero-sum multi-agent disturbance-augmented training that kept policies stable at $10\times$ worst-case perturbations.}%
% \resumeItem{Validated robustness on standard Gymnasium locomotion tasks—Ant, Humanoid, HalfCheetah, and Walker2d.}%
% \resumeItem{Created a ROS 2 hardware-in-the-loop node in C++/Python to validate RL policies on embedded hardware.}%
\resumeItem{Outperformed classical MPC, cutting trajectory-tracking error by 22 \% within strict on-board CPU/memory limits.}%
\resumeItem{Designed zero-sum, disturbance-augmented training that kept policies stable under $10\times$ worst-case perturbations.}%
% \resumeItem{Implemented PPO, SAC, TD3, and DDPG on a 15 k-LOC PyTorch/TensorFlow + Gymnasium stack for real-time embedded control.}%
% \resumeItem{Built a 15 k-LOC PyTorch/TensorFlow + Gymnasium stack for real-time control on embedded robots.}
% \resumeItem{Implemented four RL algorithms: DDPG, TD3, SAC, and PPO.}
\resumeItem{Engineered 15 k-LOC RL control stack (DDPG, TD3, SAC, PPO) in PyTorch/TensorFlow+Gym for embedded robots.}
\resumeItem{Validated robustness on Gymnasium locomotion tasks—Ant, Humanoid, HalfCheetah, Walker2d.}%
\resumeItem{Ported the system to a C++/Python ROS 2 hardware-in-the-loop node for on-board testing.}%
% \resumeItem{Engineered 15 kLOC RL control stack (DDPG|TD3|SAC|PPO) in PyTorch/TensorFlow+Gym for embedded robots and ported it to a C++/Python ROS 2 HIL node.}


% \resumeItem{Engineered a 15 k-LOC PyTorch/TensorFlow + Gym RL control stack (DDPG, TD3, SAC, PPO) and ported it to a C++/Python ROS 2 HIL node.}

% \resumeItem{Outperformed tuned MPC, cutting trajectory-tracking error 22\% within strict on-board CPU/RAM limits and validated on Ant, Humanoid, HalfCheetah, and Walker2d.}

% \resumeItem{Devised zero-sum disturbance augmentation, keeping policies stable under $10\times$ worst-case perturbations.}




% \resumeItem{Deployed as a C++/Python ROS\,2 HIL node and cut trajectory-tracking error 22\% vs.\ MPC on CPU/RAM-constrained hardware across Ant, Humanoid, HalfCheetah, and Walker2d.}

% \resumeItem{Developed zero-sum disturbance training to preserve policy stability under 10× worst-case perturbations.}




    %  \vspace{-4pt}
    % \resumeItem{Open-sourced full code, LaTeX report, and Jupyter demos to enable one-command reproduction for future researchers.}%
  \resumeItemListEnd
    \resumeSubheading
      {Researcher at
      	\href{https://www.linkedin.com/company/cnav-lab/}{\textcolor{blue}{CNAV Lab \faLinkedin}} \href{https://github.com/CNAVLAB}{\faGithub} \textcolor{red}{
          \href{https://youtube.com/@cnavlab?si=Fc-Y3oyKdgAmz3-5}{\faYoutube}}
      	 }{May\ 2020 -- Feb.\ 2025
         }
      {Head of Lab (Current), Researcher (Former)}{Tehran, Iran}
      \resumeItemListStart
        \resumeItem{Led projects on \textbf{Embedded AI} in C, \textbf{Reinforcement Learning (RL)}, and \textbf{ROS} for robotic control systems.}
        % \resumeItem{Developed \textbf{Multi-Agent} tech and AI navigation, enhancing vehicle \textbf{Precision} and \textbf{Safety}.}
        \resumeItem{Robust in-motion Transfer Alignment method based on the multilayer Neural Network. \href{https://github.com/alibaniasad1999/INS-AI}{\faGithub}
        \begin{itemize}
        \item Proposed \textbf{LSTM–MLP} that performs in-motion using only IMU + SINS data when GPS is unavailable.
        \item Cuts navigation drift to \(\!<0.1\%\) of the Kalman-INS error during 100 s GPS outages.
        \item Generalises across ship, ROV, and car datasets, outperforming conventional methods out of domain.
        \end{itemize}
        }
      \resumeItemListEnd
    %  \vspace{-4pt}
    % \resumeSubheading
    %   {Reinforcement Learning for Robotics in Complex Dynamical Systems 
    %   	{\href{https://github.com/alibaniasad1999/master-thesis}{{\normalfont \textit{\textcolor{blue}{} \faGithub  \textcolor{blue}{}
    %   	}}} }
    %   }{August 2022 -- February 2025}
    %   {Reinforcement Learning for Robotics in Complex Dynamical Systems}{Tehran, Iran}
    %   \resumeItemListStart
    %     \resumeItem{Investigated various \textbf{Reinforcement Learning} methods and compared their performance to classic control strategies.}
    %     \resumeItem{Integrated \textbf{ROS} to implement and test \textbf{Real-World Robotic} systems, validating performance in practical scenarios.}
    % \resumeItemListEnd

    \resumeSubheading
    {
    Game Theory-Based Control for Three Degrees of Freedom Platform    	{\href{https://github.com/alibaniasad1999/bachelor-thesis}{{\normalfont \faGithub 
    	}}} }
    {Feb.\ 2021 -- Sep.\ 2023
    }
    {Bachelor's Thesis, Sharif University of Technology}{Tehran, Iran}
    \resumeItemListStart
%     \resumeItem{Controlled a
%     \href{https://gcrc.sharif.edu/استند-آزمایشگاهی-کنترل-وضعیت-سه-درجه-آ/}{\textcolor{blue}{3DoF setup}}
%     using \textbf{Differential Game} theory, employing \textbf{Nash Equilibrium} for \textbf{Robust} controller.}
%     \resumeItem{Evaluated performance through Simulink simulations and practical \textbf{Implementation} on an experimental setup.}
%     \resumeItem{Modelled a 3-DoF quadcopter stand and implemented a MATLAB/Simulink–to–C pipeline for real-time testing.}%
%     \resumeItem{Designed a robust controller based on \textbf{Differential Game} theory using \textbf{Nash Equilibrium} to stabilize a 3-DoF quadcopter stand.}%
% \resumeItem{Modelled the system in Simulink and deployed it in real-time via a MATLAB/Simulink–to–C pipeline, validating performance experimentally.}%
% \resumeItem{Controlled a \href{https://gcrc.sharif.edu/استند-آزمایشگاهی-کنترل-وضعیت-سه-درجه-آ/}{\textcolor{blue}{3-DoF quadcopter stand}} using \textbf{Differential Game} theory and \textbf{Nash Equilibrium} for robust control.}%
% \resumeItem{Simulated in Simulink and deployed via a MATLAB/Simulink–to–C pipeline for real-time experimental validation.}%
% \resumeItem{Designed a robust controller based on \textbf{Differential Game} theory using \textbf{Nash Equilibrium}.}%

% \resumeItem{Simulated a \href{https://gcrc.sharif.edu/استند-آزمایشگاهی-کنترل-وضعیت-سه-درجه-آ/}{\textcolor{blue}{3-DoF quadcopter stand}} in Simulink and performed full dynamic modeling with parameter estimation for controller design.}%

% \resumeItem{Implemented designed controller  via a MATLAB/Simulink–to–C pipeline for real-time testing.}%
% \resumeItem{Benchmarked performance against \textbf{ADRC} and \textbf{DOBC}, showing improved robustness under disturbance and uncertainty.}%
\resumeItem{Modelled a \href{https://gcrc.sharif.edu/استند-آزمایشگاهی-کنترل-وضعیت-سه-درجه-آ/}{\textcolor{blue}{3-DoF setup}} in Simulink and identified dynamics for parameter estimation.}%
\resumeItem{Designed a robust controller via \textbf{Differential Game} theory and \textbf{Nash Equilibrium}.}%
\resumeItem{Implemented the controller through a MATLAB/Simulink–to–C pipeline for real-time hardware tests.}%
\resumeItem{Benchmarked against \textbf{ADRC} and \textbf{DOBC}, achieving superior disturbance rejection and robustness.}%



    \resumeItemListEnd


  


  \resumeSubHeadingListEnd
\vspace{-16pt}


% %-----------PROJECTS-----------
% \section{Projects}
%     \vspace{-8pt}
%     \resumeSubHeadingListStart

%     \resumeProjectHeading
%     {
%     	\textbf{\href{https://github.com/alibaniasad1999/Coordination-of-Multi-Agent-Autonomous-Systems}{\textcolor{blue}{Coordination of Multi-Agent Autonomous Systems}}}
%     	$|$ \emph{Embedded C, HIL, Optimization, Simulink}}{July 2023}
%     \resumeItemListStart
%     \resumeItem{Developed a multi-agent model for optimized autonomous coordination under
%     \textbf{Real-World Constraints}.}
%     \resumeItem{Implemented and validated the model with Simulink simulations and \textbf{HIL} testing using a \textbf{Microcontroller}.}
%     %		\resumeItem{Conducted field tests with a 30-drone system, demonstrating effective swarm coordination.}
%     \resumeItemListEnd

%           \vspace{-18pt}
%           \resumeProjectHeading
%           {\textbf{\href{https://github.com/alibaniasad1999/Heuristic-optimization-algorithms}{\textcolor{blue}{ Multi-Objective Heuristic Optimization}}} $|$ \emph{OOP, Optimization Algorithms, Python}}{February 2023}
%           \resumeItemListStart
%           \resumeItem{Implemented the
%           	\href{https://link.springer.com/article/10.1007/s12652-022-04332-8}{\textcolor{blue}{REMARK}}
%           	algorithm for \textbf{Multi-Objective} optimization with conflicting objectives.}
%           \resumeItem{Utilized heuristic methods to achieve high approximations of the \textbf{Pareto Set}, balancing trade-offs between objectives.}
%           \resumeItemListEnd


%           \vspace{-18pt}


%           \resumeProjectHeading
%           {
%           	\textbf{\href{https://github.com/alibaniasad1999/DATCOM-Trim-Diagram-GUI}{\textcolor{blue}{Advanced Aircraft Trim Stability Analysis}}}
%           	$|$ \emph{Advanced UI, Aircraft Control, Python}}{March 2022}
%           \resumeItemListStart
%           \resumeItem{Developed an advanced UI software, \textbf{Optimizing Analysis Processes} and enhancing design precision.}
%           %      \resumeItem{Facilitated the creation and analysis of aircraft trim diagrams, enhancing design precision.}

%           \resumeItemListEnd



%           \vspace{-18pt}
%           \resumeProjectHeading
%           {
%           	\textbf{\href{https://github.com/alibaniasad1999/Airplane-Design-II}{\textcolor{blue}{AIAA Regional Jet Design Competition}}}
%           	$|$ \emph{Aircraft Design, Computer Modeling, MATLAB, Python}}{June 2021}
%           \resumeItemListStart
%           \resumeItem{Fully designed a regional jet, encompassing \textbf{Coding}, \textbf{Computer Design}, and \textbf{Simulations}.}
%           %          \resumeItem{Showcased expertise in aircraft development through comprehensive modeling and analysis for the competition.}
%           \resumeItemListEnd


%     \resumeSubHeadingListEnd
% \vspace{-18pt}



%-----------INVOLVEMENT---------------



 \section{Awards and Honors}
 %\resumeSubHeadingListStartt
 %\resumeSubheading{Fraternity}{Spring 2020 -- Present}{President}{University Name}
%  \resumeItemListStart
%  \resumeItem{Iranian Aerospace Society’s \textbf{Best Undergraduate Thesis} Award.}
%  \vspace{-2pt}
%  \resumeItem{Ranked \textbf{Top 0.5\%} in Nationwide Undergraduate Entrance Exam among more than 150,000 participants,
%  	2017.}

% {\small
% \begin{multicols}{2}
% \begin{itemize}[itemsep=-2pt, parsep=3pt]
%   \item Best B.Sc.\ thesis award — by aerospace society.
%   \item 23\textsuperscript{rd} nationwide — 2022 Iran M.Sc.\ aerospace exam
%   \item Top 0.5 \% of 150 k (2017) in Iran’s B.Sc.\ entrance exam.
%   \item NODET exceptional-talent scholar, 2010–2017.
% \end{itemize}
% \end{multicols}
% }
\begingroup
  \small
  \begin{multicols}{2}
    \begin{itemize}[itemsep=1pt, parsep=0pt, leftmargin=*]
      \item Best B.Sc.\ Thesis Award, Iranian Aerospace Society (2023)
      \item Ranked 23rd nationally, Iran M.Sc.\ Aerospace Exam (2022)
      \item Top 0.5\% of 150\,000, Iran B.Sc.\ Entrance Exam (2017)
      \item {\href{https://en.wikipedia.org/wiki/National_Organization_for_Development_of_Exceptional_Talents}{\textcolor{blue}{NODET}}} exceptional-talent scholar. (2010–2017)
    \end{itemize}
  \end{multicols}
\endgroup

  
%  \vspace{-6pt}
%  \resumeItem{\textbf{Member of NODET} (National Organization for Development of Exceptional Talents).}
%  \resumeItemListEnd
 %\resumeSubHeadingListEnd
 \vspace{-6pt}


%
%-----------PROGRAMMING SKILLS-----------
\section{Technical Skills} %
% \begin{footnotesize}
%  \begin{itemize}[leftmargin=0.15in, label={}]
%  	\small{\item{
% \textbf{Programming Languages}{: C/C++, Embedded C, MATLAB, Python \faPython} \\
% \textbf{Tools and Platforms}{: Git \faGit*, Linux \faLinux, ROS, Simulink, \faTerminal  Terminal,  \LaTeX} \\
% \textbf{Libraries/Frameworks}{: Matplotlib, NumPy, Pandas, PyTorch, TensorFlow} \\
% % \textbf{Quantitative Skills}{: Reinforcement Learning, Robotics, Data Structures, Deep Learning, Embedded Machine Learning, Heuristic Optimization, Game Theory}

% % 			\textbf{Languages}{: Farsi (Native), English (Full Professional Proficiency)} \\
%  	}}
%  \end{itemize}
%  \end{footnotesize}


% ----------------------------
% In your document where you want the skills list:
% ----------------------------
\begingroup
  \footnotesize        % <-- switch to footnote size for everything in this group
  \begin{itemize}[leftmargin=0.15in,label={}, itemsep=-2pt, parsep=3pt] 
    \item \textbf{Programming Languages:} C/C++, Embedded C, MATLAB, Python
    \item \textbf{Tools \& Platforms:} Git, Linux \faLinux, ROS, Simulink, \faTerminal Terminal, \LaTeX
    \item \textbf{Libraries/Frameworks:} Matplotlib, NumPy, Pandas, PyTorch, TensorFlow
  \end{itemize}
\endgroup





%  \section*{Technical Skills}
%  \begin{footnotesize}
%  \noindent
%  \textbf{Languages:} C/C++, Embedded C, Python, MATLAB \quad
%  \textbf{Frameworks:} PyTorch, TensorFlow, NumPy/Pandas, Matplotlib \\[3pt]
%  \noindent
%  \textbf{Robotics \& Control:} ROS 1/2, Simulink, MPC, RL (DDPG | TD3 | SAC | PPO) \quad
%  \textbf{Tools:} Git, Linux, Docker, \LaTeX
%  \end{footnotesize}
 
%  \section*{Technical Skills}
%  \begin{footnotesize}
%  \noindent
%  \textbf{Languages:} C/C++, Embedded C, MATLAB, Python \quad
%  \textbf{Tools:} Git, Linux, ROS, Simulink, Terminal, \LaTeX \\[3pt]
%  \noindent
%  \textbf{Frameworks:} Matplotlib, NumPy, Pandas, PyTorch, TensorFlow
%  \end{footnotesize}
 
%  \vspace{-16pt}




\end{document}
